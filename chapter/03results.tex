%!TEX root = ../thesis.tex
\chapter{Results}
\label{ch:03results}

This chapter serves to present the results of our experiments.
In total, 360 individual trials were conducted across various LLM configurations to find out whether the inclusion of qualitative geographic context helps improve the navigation performance of LLMs.

\section{Overview of Dataset and Trial Execution}

All trials could be executed successfully by the methods described in chapter 2, meaning that we got a valid LLM response for each navigation task.
This does not imply that all responses were correct in terms of navigation success, but rather that we were able to collect all the necessary data for a subsequent analysis.

\begin{table}[h!]
\centering
\begin{tabular}{l c c c c}
\hline
\textbf{Group} & \textbf{\# Experiments} & \textbf{\# Successful} & \textbf{\# Failed} & \textbf{Success Rate (\%)} \\
\hline
Control Group & 180 & 36  & 144 & 20\% \\
Test Group    & 180 & 144 & 36  & 80\% \\
\hline
\textbf{Total} & \textbf{360} & \textbf{180} & \textbf{180} & \textbf{50\%} \\
\hline
\end{tabular}
\caption{Overview of experiments in control and test groups with success and failure counts.}
\end{table}

Out of the total 360 experiments, 180 were conducted in the control group without additional qualitative context and 180 in the test group with added context.
In the control group, 36 LLM responses were labeled as successful in their attempt to give correct navigation instructions, while 144 were labeled as failures.
This results in a success rate of just 20\% for the control group.

In contrast, the test group (with additional qualitative geographic context) gave 144 responses that were labeled as successful.
Although this is an increase compared to the control group, there were still 36 LLM responses that were labeled as failures.
The resulting success rate for the test group is 80\%, an increase of 60\% compared to the control group.
Although further analysis is necessary, this may give a first indication towards the effects of additional qualitative geographic context on LLM navigation performance.

In each city, we tasked three LLMs with answering the same 20 navigation tasks, resulting in 60 experiments per group and city (Table 3.2).


\begin{table}[h!]
\centering
\begin{tabular}{l c c c}
\hline
\textbf{City} & \textbf{\# LLMs} & \textbf{\# Tasks per LLM} & \textbf{\# Experiments per Group} \\
\hline
Münster & 3 & 20 & 60 \\
Hamburg & 3 & 20 & 60 \\
Vienna  & 3 & 20 & 60 \\
\hline
\textbf{Total} & \textbf{3} & \textbf{20} & \textbf{180} \\
\hline
\end{tabular}
\caption{Structure of experiments per city: three LLMs answering the same 20 navigation tasks, resulting in 60 experiments per group and city.}
\end{table}

In order to understand the performance differences across the tree test cities, we can take a look at the difference in task success rates per city (Table 3.3).
While all three cities show an increase in task success rate, the results vary slightly.
In Vienna and Münster, the success rate increased by 62\%, while in Hamburg it increased by 58\%.
In the control group, Münster had the lowest success rate at 19\%, while Vienna followed with an increase to 20\% and Hamburg had another increase by 1\% to a success rate of 21\%.
The highest overall score was achieved in Vienna under test conditions.
Here, 82\% of all navigation tasks were answered correctly by the LLMs.
The task success rates under test conditions for Hamburg and Münster were just slightly lower, with 81\% and 79\% respectively.

\begin{table}[h!]
\centering
\begin{tabular}{l c c c}
\hline
\textbf{City} & \textbf{Control Success Rate (\%)} & \textbf{Test Success Rate (\%)} & \textbf{± (Difference)} \\
\hline
Münster & 19\% & 81\% & +62\% \\
Hamburg & 21\% & 79\% & +58\% \\
Vienna  & 20\% & 82\% & +62\% \\
\hline
\textbf{Average} & \textbf{20\%} & \textbf{80.7\%} & \textbf{+60.7\%} \\
\hline
\end{tabular}
\caption{Success rates of control and test groups across the three test cities including the difference between groups.}
\end{table}


\section{Task Success Results}

\section{Statistical Significance Testing}

\section{Summary of Findings}



