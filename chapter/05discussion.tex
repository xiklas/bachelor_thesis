%!TEX root = ../thesis.tex
\chapter{Discussion}
\label{ch:05discussion}

Our study aimed to investigate the impact of qualitative geographic context on the navigation performance of LLMs.
We hypothesized that our approach would improve the ability of LLMs to provide accurate navigation instructions.
In order to test our hypothesis, we conducted a series of experiments by means of the methods described in Chapter~3.
The results of these experiments presented in Chapter~4 serve as the basis for the following discussion.

\section{Interpretation of Results}

Our experiments suggest that the inclusion of qualitative geographic context had a positive effect on LLM navigation performance in the tested setup.
The overall success rate increased from 0/120 (0\%) in the control group to 75/120 (62.5\%) in the test group (see Table~4.2).
This is a substantial improvement in task success rate, supporting our initial research hypothesis in the within the scope of our experiments.
However, since the evaluation in this thesis does not include formal statistical significance testing, these results should be interpreted as empirical evidence rather than a general claim about all possible cities, models, or navigation scenarios.

To reduce the risk that results depend on a specific city-model combination, we analyzed multiple cities and LLMs.
Across all of the tested combinations, we observed improvements in trial success rates when qualitative geographic context was provide (see Table~4.3).
This suggests a potential generalizable effect of our approach, although further research beyond the scope of this thesis would be necessary to confirm this effect.
Although we tested six different city-model combinations in total (2 cities and 3 models), and saw improvements across all of them, it is important to state that this is just a small fraction of the possible combinations of cities and LLMs that exist.

The causes for the poor control group performance remain speculative.
Although the reasoning capabilities of LLMs have been shown to be impressive in other areas, one possible explanation is that these are not sufficient for navigation task yet.
Another possible cause is that there may be a lack of geographic knowledge in the training data of LLMs. 
After all, traditional navigation solutions usually do not rely on qualitative spatial data published on the internet, but rather on large databases of geospatial datasets.
Our approach would thus simply provide the LLMs with the information they need to make up for this blind spot in the training data.
Although this may be a sensible explanation, our research cannot confirm this assumption, and further studies would be necessary to answer this tangential question.

Our results indicate that poor LLM navigation performance is not inevitable however.
Instead, by providing qualitative geographic context, the results of navigation performance could be improved substantially.
This serves as yet another example of LLMs and their many different abilities, but does this mean that traditional navigation solutions will become obsolete and replaced by LLMs in the future?
While it has become easier to imagine LLMs being used for navigation tasks, as we have shown that their performance could be improved measurably, in our view it is irresponsible to make such bold claims.

This study does not claim that traditional navigation solutions are no longer necessary or that LLMs are going to replace them.
In fact, suggesting such a thing would feel irresponsible:
The many mistakes made by the LLMs in the control group as well as the mistakes still present in the test group highlight the current limitations of LLM-based navigation and the need for additional research.

Mistakes in navigation can lead to serious consequences in time, safety and cost in real world scenarios.
Although certainly encouraging, a success rate of 62.5\% is therefore not sufficient for many use cases, while established navigation systems typically achieve high reliability in everyday use.
If the gap could be closed further in the future, LLMs could someday seriously be considered as alternatives to current navigation solutions.

So far, we have only considered topological information when providing geographic context to LLMs.
In practice, there are countless other types of geographic information that could be supplied for various use cases as well.
Future work could explore the inclusion of additional types of geographic context:
accessibility information, data on points of interest or data on real-time dynamic conditions could potentially be rich avenous.

The development of new models is rapid and ongoing:
areas of poor performance today may be addressed by a new model tomorrow.
This rapid model progress may reduce the marginal benefit of context enrichment over time, motivating repeated evaluation as new models are released.

Research on LLM-based navigation could also benefit from improvements in validation techniques.
Manual validation as done in this study does not scale well and is not suitable for larger datasets or the coverage of many models.
Eventually, manual validation may become unrealistic as it becomes too time-consuming and labor-intensive.
Automated validation methods could help address this issue, and also help the reproducibility of results.

In this study, we demonstrated one possible way to improve LLM navigation performance.
Although the results were promising, this doesn't mean that other methods could not achieve similar or even better results than ours.
We encourage researchers to explore the development of other alternative methods.
The testing framework described in Chapter~3 was designed to be easily reproducible and could serve as a basis for future research. 

Beyond navigation, our approach could be used in similar scenarios involving graph like structures.
One interesting avenue for additional research could thus be to study whether knowledge graphs could be represented using qualitative relations as well, and whether this could allow LLMs to reason about knowledge graphs in a similar way as we have shown for geographic networks.

Our results can also be interpreted in another way: If it takes additional, non-trivial work to achieve better navigation performance with LLMs, maybe they are currently not the best tool for the job after all.
With our context enrichment setup, we were able to achieve higher trial success rates in our test group compared to the low success rate of the control group.
At present, implementing a comparable pipeline in user navigation applications could prove to be non-trivial however.
It is difficult to imagine users, especially those without a technical background, recreating our setup on mobile devices such as smartphones or smartwatches for example, which are so commonly used for navigation in this era of mobile computing.

Following this interpretation, LLM systems may benefit from further calibration, and explicit caveats about their limitations when answering user's navigation queries could be an appropriate measure.
Our control group experiments, did not indicate this in present implementations of LLMs, however.
Many mistakes of various nature were made by the tested LLMs, even though they are widely used state-of-the-art models at the time of writing.
And since these models used so frequently today, it is not implausible that some users have already tried to navigate using them.

It would be interesting to know whether users would be aware of the potentially poor results.
In the best case, users would quickly realize the mistakes before following them and move to other existing navigation solutions to get accurate navigation results.
Even then however, this could lead to undermined trust in LLMs in general, which is certainly not to be desired by their providers.
In adverse cases, users could face delays or safety-relevant situations if they were to follow incorrect navigation instructions provided by LLMs.

Overall, the approach presented in this thesis is promising, but a performance gap to traditional navigation solutions as well as other unanswered questions remain.
While there are use cases for LLMs in combination with geospatial data, careful evaluation remains a necessity when coming up with new ways of applying LLMs to problems in real-world scenarios.

This study should thus be interpreted as a proof of concept:
LLM navigation performance can be improved by providing qualitative geographic context.
This does not suggest however that LLMs are ready replace traditional navigation solutions yet, but that does not mean that the idea  of LLM based navigation can be fully dismissed either.

\section{Limitations}

While our methods have shown promising results, there are limitations.
First, our study considered only two distinct areas of interest within just two European cities.
Although we observed improvements in both cities, it is possible that other cities may perform worse using our approach.
Data quality can also differ when considering different regions.
While OpenStreetMap coverage was sufficient for our selected cities, this may not be the case in other cities or regions.
Thus, generalization to other cities or regions remains an open question.

Second, we only evaluated three LLMs in this study.
Today, there are endless different LLMs available, and it is likely that many more will be released in the future.
While we were able to observe improvements across all three tested models, this does not guarantee that performance improvements will occur with any LLM.
Results may also vary with future architectures, augmentation techniques or model versions.