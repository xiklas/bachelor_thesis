%!TEX root = ../thesis.tex
\chapter{Discussion}
\label{ch:05discussion}

Our study aimed to investigate the impact of qualitative geographic context on the navigation performance of LLMs.
We hypothesized that our approach would improve the ability of LLMs to provide accurate navigation instructions.
In order to test our hypothesis, we conducted a series of experiments by means of the methods described in Chapter~3.
The results of these experiments presented in Chapter~4 serve as the basis for the following discussion.

\section{Interpretation of Results}

Our experiments revealed that the inclusion of qualitative geographic context had a positive effect on LLM navigation performance.
The overall success rate increased from 0/120 (0\%) in the control group to 75/120 (62.5\%) in the test group (see Table~4.2).
This is a substantial improvement in task success rate, and thus our experiments support our initial research hypothesis within the scope of the tested setup.
However, since the evaluation in this thesis does not include formal statistical significance testing, these results should be interpreted as empirical evidence rather than a general claim about all possible cities, models, or navigation scenarios.

To reduce the risk that results depend on a specific city-model combination, we analyzed multiple cities and LLMs.
Across all of the tested combinations, we observed similar levels of improvement.
This leads us to believe that there may be a generalizable effect of qualitative geographic context on LLM navigation performance, as long as the models possess sufficient reasoning capabilities, which may be needed to process the additional context effectively.
Although we tested six different city-model combinations in total (2 cities and 3 models), and saw improvements across all of them, it is important to state that this is just a small fraction of the possible combinations of cities and LLMs that exist.

The reasons for the rather lackluster LLM performance in the control group still remain unknown to us.
Since the reasoning capabilities of LLMs have been shown to be impressive in other areas, we suspect it is unlikely that the inherent abilities of the models are the limiting factor for the poor navigation performance.
A possible cause is the absence of qualitative geographic training data in the datasets used to train these models.
LLMs are trained on large amounts of text data from the internet, which may simply not contain sufficient amounts of descriptions of spatial relationships in the format LLMs can understand most effectively - text.
After all, traditional navigation solutions usually do not rely on qualitative spatial data published on the internet, but rather on large databases of geospatial datasets.
Our approach would thus simply provide the LLMs with the information they need to make up for this blind spot in the training data.
Although this may be a sensible explanation, our research cannot confirm this assumption, and further studies would be necessary to answer this tangential question.

We have shown that the poor navigation performance of LLMs is not inevitable. 
Instead, by means of simply providing qualitative geographic context, the results of navigation performance could be improved significantly.
This serves as yet another example of LLMs and their many different abilities, but does this mean that traditional navigation solutions will become obsolete and replaced by LLMs in the future?
While it has become easier to imagine LLMs being used for navigation tasks, as we have shown that their performance could be improved measurably, in our view it is irresponsible to make such bold claims.

This study does not intend to suggest that traditional navigation solutions are inferior and should be labeled obsolete anytime soon.
In fact, suggesting such a thing would feel irresponsible:
The many mistakes made by the LLMs in the control group as well as the mistakes still present in the test group highlight the current limitations of LLM-based navigation and the need for additional research.

Navigation is often critical, and any mistakes can have serious consequences in terms of time, safety or efficiency.
Although certainly encouraging, a success rate of 62.5\% is therefore not sufficient for many use cases, when traditional navigation solutions operate at near perfection.
If the gap could be closed further in the future, LLMs could someday seriously be considered as alternatives to current navigation solutions.

So far, we have only considered topological information when providing geographic context to LLMs.
In practice, there are countless other types of geographic information that could be supplied for various use cases as well.
Perhaps the inclusion of accessibility information could aid people with disabilities.
Maybe information on current events could help users make informed decisions on their route depending on the situation.

The development of new models is rapid and there is no end in sight.
Areas of poor performance today may be addressed by a new model tomorrow.
This may render parts of our research obsolete in the future.
However, our overall approach and the framework of integrating qualitative context has been proven to work at least in this area and could see success in various other domains, although there is no way to tell that the approach generalizes well just by looking at the results of this study.

Further research on the validation of LLM performance may also aid in the development of LLM-based navigation.
Validating hundreds of navigation tasks manually as was done in this study is time-consuming and impractical for larger datasets.
Automated validation methods could help address this issue and provide further evidence for the effectiveness of the technique.

In this study, we demonstrated one possible way to improve LLM navigation performance.
Although the results were promising, this doesn't mean that other methods could not achieve similar or even better results than ours.
We encourage researchers to try their hand at coming up with alternative methods.
Our easily reproducible testing framework described in Chapter~3 may be used as basis for such future studies.

If we look beyond the scope of navigation tasks, our findings may also have relevance for other areas in which LLMs could be combined with graph data.
Knowledge graphs are used in many domains and have a similar makeup to geographic networks.
One interesting avenue for additional research could thus be to study whether knowledge graphs could be represented using qualitative relations as well, and whether this could allow LLMs to reason about knowledge graphs in a similar way as we have shown for geographic networks.

Our results can also be interpreted in another way: If it takes additional, non-trivial work to achieve good navigation performance with LLMs, maybe they are currently not the best tool for the job after all.
With our context enrichment setup, we were able to achieve good results in a testing environment.
This setup is however not practical to implement in all real-world scenarios yet.
It is difficult to imagine users, especially those without a technical background, recreating our setup on mobile devices such as smartphones or smartwatches for example, which are so commonly used for navigation in this era of mobile computing.

Following this interpretation, LLMs should inform users that they are incapable of delivering accurate answers to navigation tasks.
In our control group experiments, we have seen that this is not the case, however.
Many incorrect street names, impossible connections and other mistakes were made by all models, even though they are among the most advanced models available to the public at the time of writing.
And since these models are so widely used today, it is not unlikely that many users have already tried to navigate using them.

It would be interesting to know whether users would be aware of the potentially poor results.
In the best case, users would quickly realize the mistakes before following them and move to other existing navigation solutions to get accurate navigation results.
Even then however, this could lead to undermined trust in LLMs in general, which is certainly not to be desired by their providers.
In the worst case, users could blindly follow the incorrect instructions, get lost, and consequently lose time - or worse.

Additionally, the climate impact of LLM usage has been an ongoing topic of discussion.
Knowing more about LLM capabilities and limitations could help users make conscious decision about when to employ an LLM and when to rely on other tools.

We hope that this discussion of our results made clear that while our approach looks promising, we do not want to jump to extreme conclusions.
There are use cases for LLMs in combination with geospatial data.
Our approach has shown that the use case of navigation could certainly be made more reliable.
But yet there is still a gap between the performance of our setup and traditional navigation solutions.

The steady improvement in LLM systems can spark belief into a future where LLMs can solve navigation tasks as well as any other existing solution, but we feel that it is important to stay grounded and not overhype the latest technologies and become blind to their obvious shortcomings.

Some recent products tried to make LLM-based AI a key selling point, only to underperform and be met with harsh user criticism.
Although these failures were not mainly in the area of navigation, and the reasons for the product failures are certainly manifold, they still serve as examples of the dangers of overhyping LLM based solutions for just any use case.

This study should thus be interpreted as a first step, a proof of concept that LLM navigation can be improved.
It doesn't intend to show that LLMs are ready to be a viable alternative to traditional navigation solutions yet, but that doesn't mean that the idea can be fully dismissed either.

\section{Limitations}

While our methods have shown promising results, there are limitations.
First, our study considered only two distinct areas of interest within just two European cities.
Although we observed improvements in both cities, it is possible that other cities may perform worse using our approach.
Data quality can also differ when considering different regions.
While OpenStreetMap provided good coverage for our selected cities, this may not be the case everywhere and worse data quality could ultimately lead to worse navigation performance using our approach.

Second, we only evaluated three LLMs in this study.
Today, there are endless different LLMs available, and it is likely that many more will be released in the future.
While we were also able to observe improvements across all three tested models, this may not be the case for all models available.
