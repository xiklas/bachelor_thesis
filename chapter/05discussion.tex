%!TEX root = ../thesis.tex
\chapter{Discussion}
\label{ch:05discussion}

Our study aimed to investigate the impact of qualitative geographic context on the navigation performance of LLMs.
We hypothesized that our approach would improve the ability of LLMs to provide accurate navigation instructions.
In order to test our hypothesis, we conducted a series of experiments by means of the methods described in Chapter~3.
The results of these experiments presented in Chapter~4 serve as the basis for the following discussion.

\section{Interpretation of Results}

Our experiments suggest that the inclusion of qualitative geographic context had a positive effect on LLM navigation performance in the tested setup.
The overall success rate increased from 0/120 (0\%) in the control group to 75/120 (62.5\%) in the test group (see Table~4.2).
This is a substantial improvement in task success rate, supporting our initial research hypothesis within the scope of our experiments.
However, since the evaluation in this thesis does not include formal statistical significance testing, these results should be interpreted as empirical evidence rather than a general claim about all possible cities, models, or navigation scenarios.

The success rate of 0\% in the control group is striking, and raises multiple questions.
First: was the experimental setup appropriate to test LLM navigation performance in real world scenarios?
Since the navigation tasks were designed by combining two real world streets in each city with each other to form a routing task, the setup realistically reflects potential real world user scenarios.
Additionally, the sole use of LLMs for navigation without any additional tools may be questioned.
However, since a goal of this study was to evaluate LLM navigation performance in the first place, this setup was necessary to isolate the LLMs' capabilities from any external sources or information.
Further, our correctness criteria could be considered too strict by some.
This does not align with typical navigation use cases however, where a single wrong instruction can lead to failed navigation, regardless of the system used.
Under this evaluation framework, the 0\% success rate indicates that the tested LLMs were not reliable for navigation in the tested areas.

Further, as depicted in Table~4.1, the performance increase across the two tested cities was not equal.
In Hamburg, the success rate in the test group was 86.6\%, while in Münster it was only 38.3\%.
Several factors could explain this discrepancy.
One possible explanation is the size difference between the two tested datasets.
The Hamburg dataset contained just 38 individual streets, while the Münster dataset contained 128 individual streets.
Perhaps the smaller dataset was easier for the LLM to process in its context window, leading to better performance.
Additionally, the navigation tasks in Münster spanned a larger distance (1272m) compared to Hamburg (899m), as measured using Google Maps.
Thus, another possible explanation is that the LLMs struggled to find effective solutions for these longer routes.
Since the control group performance was equally poor in both cities, it seems unlikely that any intrinsic model bias towards one of the cities played a role in the observed results.
However, such biases cannot be entirely ruled out without further experiments.
In total, it may be possible that multiple factors contributed to the observed performance difference between the two cities.

We may also consider the performance differences between the three tested LLMs shown in Figure~4.2 and discuss whether it made sense to test multiple models, and whether the choice of models was appropriate.
The results indicate that while all three models benefited from the provided geographic context, their performance varied in the tested setup.
This suggests that navigation performance using our approach may depend on the specific LLM used, justifying our decision to test multiple models.
While the exact reasons for the performance differences remain speculative, differences in model architecture, training process, or fine tuning could have played a role.
The best performing model in our experiments was Gemini 2.5 Pro.
This does not imply however, that it is the best model using our approach in general: other models not evaluated in this study may perform better or worse than our tested models.
However, the substantial increase in success rates across all three models indicates that our approach has the potential to improve navigation performance in a variety of current LLMs, which could be an area of interest for future research.

In Chapter~2 we discussed several LLM related benchmarks.
This raises the question whether our framework could be integrated or expand on existing benchmarks.
As we have seen, all models in the control group had a success rate of 0\%.
Therefore, it seems unlikely that our framework would provide interesting insights when purely testing LLM performance without any additional techniques to boost navigation performance. 
For any techniques aiming to improve LLM navigation however, our framework could serve as a navigation-specific task suite to compare performance improvements.

We also discussed the suitability of OpenStreetMap data for our approach in Chapter~2.
In this thesis, OSM data was used to extract street networks for Hamburg and Münster.
Whether variance in OSM data quality could influence our results was not explicitly tested in this study and remains speculative.
Issues in OSM data quality are unlikely to be the main cause for all test group mistakes however, since the provided context was sufficient to improve results across all tested models and cities.
Future research could explore different data sources in order to better understand the influence of data sources on LLM based navigation.

In addition, we may consider the suitability of the dipole relations presented in Chapter~2 for our approach.
In this study, these relations were used to represent geographic context in a qualitative manner.
The results indicate that this representation was effective:
While the previously discussed variations in performance across cities and models cannot be ignored, the overall improvements show that dipole relations can be a useful tool for LLM navigation tasks.
Whether alternative frameworks for spatial representation could yield comparable or even better results remains an open question for future research.

Our results indicate that poor LLM navigation performance is not necessarily inevitable in the tested setup.
By providing qualitative geographic context, navigation performance improved substantially in our experiments.
This indicates that LLM navigation performance can be improved with additional techniques.
In practice, this does not imply that established navigation solutions are obsolete.
Based on the results of this study, no claim of LLM based navigation entirely replacing established navigation solutions can be made.
The many mistakes made by the LLMs in the control group as well as the mistakes still present in the test group highlight the current limitations of LLM-based navigation and justify alternative solutions.

Mistakes in navigation can lead to serious consequences in time, safety and cost in real world scenarios.
Although certainly encouraging, a success rate of 62.5\% is therefore not sufficient for many use cases, while established navigation systems typically achieve high reliability in everyday use.
If the gap could be closed further in the future, LLMs could someday seriously be considered as alternatives to current navigation solutions.

So far, we have only considered topological information when providing geographic context to LLMs.
In practice, there are countless other types of geographic information that could be supplied for various use cases as well.
Future work could explore the inclusion of additional types of geographic context:
accessibility information, data on points of interest or data on real-time dynamic conditions could potentially be rich avenues.

The development of new models is rapid and ongoing:
areas of poor performance today may be addressed by a new model tomorrow.
This rapid model progress may reduce the marginal benefit of context enrichment over time, motivating repeated evaluation as new models are released.

Research on LLM-based navigation could also benefit from improvements in validation techniques.
Manual validation as done in this study does not scale well and is not suitable for larger datasets or the coverage of many models.
Eventually, manual validation may become unrealistic as it becomes too time-consuming and labor-intensive.
Automated validation methods could help address this issue, and also help the reproducibility of results.

In this study, we demonstrated one possible way to improve LLM navigation performance.
Although the results were promising, this doesn't mean that other methods could not achieve similar or even better results than ours.
We encourage researchers to explore the development of other alternative methods.
The testing framework described in Chapter~3 was designed to be easily reproducible and could serve as a basis for future research. 

Beyond navigation, our approach could be used in similar scenarios involving graph-like structures.
One interesting avenue for additional research could thus be to study whether knowledge graphs could be represented using qualitative relations as well, and whether this could allow LLMs to reason about knowledge graphs in a similar way as we have shown for geographic networks.

Our results can also be interpreted in another way: If it takes additional, non-trivial work to achieve better navigation performance with LLMs, maybe they are currently not the best tool for the job after all.
With our context enrichment setup, we were able to achieve higher trial success rates in our test group compared to the low success rate of the control group.
At present, implementing a comparable pipeline in user navigation applications could prove to be non-trivial however.
It is difficult to imagine users, especially those without a technical background, recreating our setup on mobile devices such as smartphones or smartwatches for example, which are so commonly used for navigation in this era of mobile computing.

Following this interpretation, LLM systems may benefit from further calibration, and explicit caveats about their limitations when answering users' navigation queries could be an appropriate measure.
Our control group experiments did not indicate this in present implementations of LLMs, however.
Many mistakes of various nature were made by the tested LLMs, even though they are widely used state-of-the-art models at the time of writing.
And since these models are used so frequently today, it is not implausible that some users have already tried to navigate using them.

It would be interesting to know whether users would be aware of the potentially poor results.
In the best case, users would quickly realize the mistakes before following them and move to other existing navigation solutions to get accurate navigation results.
Even then however, this could lead to undermined trust in LLMs in general, which is certainly not to be desired by their providers.
In adverse cases, users could face delays or safety-relevant situations if they were to follow incorrect navigation instructions provided by LLMs.

Overall, the approach presented in this thesis is promising, but a performance gap to traditional navigation solutions as well as other unanswered questions remain.
While there are use cases for LLMs in combination with geospatial data, careful evaluation remains a necessity when coming up with new ways of applying LLMs to problems in real-world scenarios.

This study should thus be interpreted as a proof of concept:
LLM navigation performance can be improved by providing qualitative geographic context.
This does not suggest however that LLMs are ready to replace traditional navigation solutions yet, but that does not mean that the idea of LLM-based navigation can be fully dismissed either.

\section{Limitations}

While our methods have shown promising results, there are limitations.
First, our study considered only two distinct areas of interest within just two European cities.
Although we observed improvements in both cities, it is possible that other cities may perform worse using our approach.
Data quality can also differ when considering different regions.
While OpenStreetMap coverage was sufficient for our selected cities, this may not be the case in other cities or regions.
Thus, generalization to other cities or regions remains an open question.

Second, we only evaluated three LLMs in this study.
Today, many different LLMs are available, and it is likely that many more will be released in the future.
While we were able to observe improvements across all three tested models, this does not guarantee that performance improvements will occur with any LLM.
Results may also vary with future architectures, augmentation techniques or model versions.