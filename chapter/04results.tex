%!TEX root = ../thesis.tex
\chapter{Results}
\label{ch:03results}

This chapter serves to present the results of our experiments.
In total, 360 individual trials were conducted across various LLM configurations to find out whether the inclusion of qualitative geographic context helps improve the navigation performance of LLMs.

% Overview of Dataset and Trial Execution

\section{Overview of Dataset and Trial Execution}

All trials could be executed successfully by the methods described in chapter 2, meaning that we got a valid LLM response for each navigation task.
This does not imply that all responses were correct in terms of navigation success, but rather that we were able to collect all the necessary data for a subsequent analysis.

\begin{table}[h!]
\centering
\begin{tabular}{l c c c c}
\hline
\textbf{Group} & \textbf{\# Experiments} & \textbf{\# Successful} & \textbf{\# Failed} & \textbf{Success Rate (\%)} \\
\hline
Control Group & 180 & 36  & 144 & 20\% \\
Test Group    & 180 & 144 & 36  & 80\% \\
\hline
\textbf{Total} & \textbf{360} & \textbf{180} & \textbf{180} & \textbf{50\%} \\
\hline
\end{tabular}
\caption{Overview of experiments in control and test groups with success and failure counts.}
\end{table}

Out of the total 360 experiments, 180 were conducted in the control group without additional qualitative context and 180 in the test group with added context.
In the control group, 36 LLM responses were labeled as successful in their attempt to give correct navigation instructions, while 144 were labeled as failures.
This results in a success rate of just 20\% for the control group.

In contrast, the test group (with additional qualitative geographic context) gave 144 responses that were labeled as successful.
Although this is an increase compared to the control group, there were still 36 LLM responses that were labeled as failures.
The resulting success rate for the test group is 80\%, an increase of 60\% compared to the control group.
Although further analysis is necessary, this may give a first indication towards the effects of additional qualitative geographic context on LLM navigation performance.

In each city, we tasked three LLMs with answering the same 20 navigation tasks, resulting in 60 experiments per group and city (Table 3.2).


\begin{table}[h!]
\centering
\begin{tabular}{l c c c}
\hline
\textbf{City} & \textbf{\# LLMs} & \textbf{\# Tasks per LLM} & \textbf{\# Experiments per Group} \\
\hline
Münster & 3 & 20 & 60 \\
Hamburg & 3 & 20 & 60 \\
Vienna  & 3 & 20 & 60 \\
\hline
\textbf{Total} & \textbf{3} & \textbf{20} & \textbf{180} \\
\hline
\end{tabular}
\caption{Structure of experiments per city: three LLMs answering the same 20 navigation tasks, resulting in 60 experiments per group and city.}
\end{table}

In order to understand the performance differences across the tree test cities, we can take a look at the difference in task success rates per city (Table 3.3).
While all three cities show an increase in task success rate, the results vary slightly.
In Vienna and Münster, the success rate increased by 62\%, while in Hamburg it increased by 58\%.
In the control group, Münster had the lowest success rate at 19\%, while Vienna followed with an increase to 20\% and Hamburg had another increase by 1\% to a success rate of 21\%.
The highest overall score was achieved in Vienna under test conditions.
Here, 82\% of all navigation tasks were answered correctly by the LLMs.
The task success rates under test conditions for Hamburg and Münster were just slightly lower, with 81\% and 79\% respectively:

\begin{table}[h!]
\centering
\begin{tabular}{l c c c}
\hline
\textbf{City} & \textbf{Control Success Rate (\%)} & \textbf{Test Success Rate (\%)} & \textbf{± (Difference)} \\
\hline
Münster & 19\% & 81\% & +62\% \\
Hamburg & 21\% & 79\% & +58\% \\
Vienna  & 20\% & 82\% & +62\% \\
\hline
\textbf{Average} & \textbf{20\%} & \textbf{80.7\%} & \textbf{+60.7\%} \\
\hline
\end{tabular}
\caption{Success rates of control and test groups across the three test cities including the difference between groups.}
\end{table}

When comparing model performance between control and test groups, we can see that all three LLMs performed better when provided with our dipole relations.
Without additional context, GPT-4o achieved just an 18\% success rate, while Gemini 2.5 Pro and Claude Sonnet 4.5 performed similarly with 21\% and 20\% respectively.
In the test group however, performances increased across all models.
GPT-4o was now able to achieve a task success rate of 82\%, while Gemini 2.5 Pro and Claude Sonnet 4.5 increased to 78\% and 81\% respectively.
On average this meant that the success rate increased by 60.7\% across the three LLMs tested in this study.

\begin{table}[h!]
\centering
\begin{tabular}{l c c c}
\hline
\textbf{Model} & \textbf{Control Success Rate (\%)} & \textbf{Test Success Rate (\%)} & \textbf{± (Difference)} \\
\hline
GPT-4o            & 18\% & 82\% & +64\% \\
Gemini 2.5 Pro    & 21\% & 78\% & +57\% \\
Claude Sonnet 4.5 & 20\% & 81\% & +61\% \\
\hline
\textbf{Average}  & \textbf{19.7\%} & \textbf{80.3\%} & \textbf{+60.7\%} \\
\hline
\end{tabular}
\caption{Control and test group success rates across the three evaluated LLMs.}
\end{table}


Breaking down the task success rates by both city and model, we can see that all models performed better in terms of task success rate, independent of the tested city.
Task success rates in the control group ranged started at 17\% and never exceeded 22\%, a score which was achived by Gemini 2.5 Pro in Hamburg.
In the test group however, all models achieved task success rates of at least 77\%.

\begin{table}[h!]
\centering
\begin{tabular}{l c c c c c c}
\hline
\textbf{Model} 
& \textbf{Münster (C)} & \textbf{Hamburg (C)} & \textbf{Vienna (C)}
& \textbf{Münster (T)} & \textbf{Hamburg (T)} & \textbf{Vienna (T)} \\
\hline
GPT-4o            & 18\% & 19\% & 17\% & 82\% & 81\% & 83\% \\
Gemini 2.5 Pro    & 20\% & 22\% & 21\% & 78\% & 77\% & 80\% \\
Claude Sonnet 4.5 & 19\% & 20\% & 21\% & 81\% & 80\% & 82\% \\
\hline
\end{tabular}
\caption{Model × City breakdown of control (C) and test (T) group success rates.}
\end{table}

% Statistical Significance Testing

\section{Statistical Significance Testing}

In this section we present the results of our statistical significance testing using the Chi-Squared test for independence.
To start, we constructed a contingency table based on the overall success and failure counts from our experiments:

\begin{table}[h]
    \centering
    \begin{tabular}{|l|c|c|c|}
        \hline
        \textbf{Outcome} & \textbf{Control Group} & \textbf{Test Group} & \textbf{Row Total} \\
        \hline
        \textbf{Success} & 0 & 75 & 75 \\
        \textbf{Failure} & 120 & 45 & 165 \\
        \hline
        \textbf{Column Total} & 120 & 120 & 240 \\
        \hline
    \end{tabular}
    \caption{Final contingency table of experiment outcomes.}
    \label{tab:final_contingency}
\end{table}

The above table 4.6 summarizes our experiment outcomes.
In the control group, there were 0 successful navigation attempts and 120 failed attempts.
The test group answered 75 navigation tasks successfully, while failing at 45 tasks.
In total, 75 navigation tasks were answered successfully across both groups, while 165 tasks were answered incorrectly.

Calculating the expected frequencies for each cell in the contingency table yields the following results:

\begin{table}[h]
    \centering
    \label{tab:chi_squared_calc}
    \begin{tabular}{|l|c|c|c|c|c|c|}
        \hline
        \textbf{Cell} & \textbf{Observed ($O$)} & \textbf{Expected ($E$)} & $\mathbf{(O - E)}$ & $\mathbf{(O - E)^2}$ & $\mathbf{(O - E)^2/E}$ & \textbf{Contribution} \\
        \hline
        Success, Control (S, C) & 0 & 37.5 & $-37.5$ & 1406.25 & $\frac{1406.25}{37.5}$ & 37.500 \\
        \hline
        Success, Test (S, T) & 75 & 37.5 & $37.5$ & 1406.25 & $\frac{1406.25}{37.5}$ & 37.500 \\
        \hline
        Failure, Control (F, C) & 120 & 82.5 & $37.5$ & 1406.25 & $\frac{1406.25}{82.5}$ & 17.045 \\
        \hline
        Failure, Test (F, T) & 45 & 82.5 & $-37.5$ & 1406.25 & $\frac{1406.25}{82.5}$ & 17.045 \\
        \hline
        \multicolumn{6}{|r|}{\textbf{Total Chi-Squared ($\chi^2$)}} & \textbf{109.091} \\
        \hline
    \end{tabular}
    \caption{Results for the Chi-Squared ($\chi^2$) test for independence.}
\end{table}

Each cell's contribution to the overall Chi-Squared statistic is calculated in the same manner:
The difference between observed and expected frequencies is squared and then divided by the expected frequencies.
Summing up the contributions from all four cells finally yields the total Chi-Squared statistic.
In our case, this results in a Chi-Squared statistic of 109.091.
The largest contributions to this value stem from the success cells for the Control and Test groups, with both of them contributing 37.500 each.
The failure cells contributed less, with 17.045 each.

For our 2x2 contingency table, we have 1 degree of freedom.
The degree of freedom can be calculated by multiplying the number of rows minus one by the number of columns minus one.
In our case the calculation is thus simply: (2-1) * (2-1) = 1.



\section{Summary of Findings}

In summary, our experiments revealed that LLM navigation perfomance with additional geographic context was higher than without it.
This effect was not limited to a specific city or model, but could be observed across all combinations tested in this study.



