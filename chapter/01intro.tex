%!TEX root = ../thesis.tex
\chapter{Introduction}
\label{ch:01intro}

% Chapter Introduction

In this chapter, we introduce the research topic of navigation using large language models.
After briefly discussing the current behavior of LLMs in the evaluated setting, we discuss the existing body of research in this area.
We then identify the research gap we aim to address in this thesis and formulate a corresponding research hypothesis.

% The Current State of Navigation using LLMs

\section{The Current State of Navigation using LLMs}

% Introduction to Research Topic

Today, users rely on large language models every day, with leading systems like ChatGPT processing over 2 billion user queries per day as of July 2025 \citep{openai_unlocking_2025}.
A recent journal article has highlighted the many unique use cases for LLMs, ranging from information retrieval to various other tasks such as drafting text or generating computer code \citep{chatterji_how_2025}.
Given their widespread use, LLMs may already be used for navigational queries in practice.
This motivates evaluating whether LLMs are currently capable of delivering accurate and reliable responses to user navigation queries. 
If performance can be shown to be insufficient, this motivates research exploring techniques to improve LLM navigation performance. 

In the following example depicted in Figure 1.1, we prompted GPT-4o with a simple navigation task to observe whether the model could successfully provide a useful answer to a simple path finding problem.
The prompt used to generate the response shown in Figure 1.1 was as follows:

\begin{quote}
    \textit{I just arrived at Münster central station and need to get to the Hafenweg. Please provide me with step-by-step walking directions to get there.}
\end{quote}

% Example Figure of User Navigation Task Reply using ChatGPT 4o

\begin{figure}[h!]
    \centering
    \includegraphics[width=0.8\textwidth]{chapter/graphics/gpt_response.png} % Adjust width as needed
    \caption{GPT-4o response to a simple route planning task.}
    \label{fig:graphic1}
\end{figure}

% Map with Correct Route vs. GPT-4o Response

\begin{figure}[h!]
    \centering
    \includegraphics[width=0.8\textwidth]{chapter/graphics/google_maps_route.png} % Adjust width as needed
    \caption{One possible correct route (green) for the example navigation task compared to various segments found in the GPT-4o response (red).}
    \label{fig:graphic2}
\end{figure}

% What is wrong with the LLM response?

Even though the model response shown in Figure 1.1 appears structurally sound and plausible at first glance, upon closer inspection there are several mistakes that would likely prevent the user from following the proposed route.
To understand the mistakes, we analyze the response step by step, starting with the first instruction.
To ensure followability, we provide Figure 1.2 showing one possible correct route as context.

\begin{quote}
    \textit{1. Exit Münster Hauptbahnhof through the main entrance onto Willy-Brandt-Allee.}
\end{quote}

Although exiting Münster Hauptbahnhof to reach the destination is certainly a sensible step, a more direct choice would be to use the secondary exit away from the city center and towards the actual destination.
Also, the response introduces a street called ``Willy-Brandt-Allee'' which does not align with the map context shown in Figure 1.2.
The second instruction is as follows:

\begin{quote}
    \textit{2. Turn right and walk southeast on Willy-Brandt-Allee toward the city center.}
\end{quote}

In the example, the destination ``Hafenweg'' lies generally southeast of the starting point.
The direction ``toward the city center'' is also inconsistent with the intended destination.
The next instruction reads:

\begin{quote}
    \textit{3. After about 500 meters, you'll reach Albersloher Weg. Turn left onto Albersloher Weg.}
\end{quote}
 
Next, the LLM response introduces the street ``Albersloher Weg''.
However, the map shown in Figure 1.2 suggests no visible connection between ``Willy-Brandt-Allee'' and ``Albersloher Weg''.
Continuing with the next instruction:

\begin{quote}
    \textit{4. Continue along Albersloher Weg for about 1 km, passing the Aasee park area on your right.}
\end{quote}

This instruction raises many questions. 
For once, the destination does not lie one kilometer away from the Central Station, therefore the stated distance appears inconsistent with the approximate start-destination distance in this example.
Further, the response introduces the ``Aasee park area'', which is inconsistent with the route context shown in Figure 1.2
Additionally, the map shown in Figure 1.2 shows no visible connection between the two segments.
With the next instruction, the mistakes made by GPT-4o continue:

\begin{quote}
    \textit{5. When you reach the intersection with Hammer Straße, turn right.}
\end{quote}

The response then refers to ``Hammer Straße''.
Again, this instruction does not appear viable in the example situation, as Figure 1.2 does not show a visible connection between the two required streets. 
Thus, it doesn't appear that this instruction could be followed in practice.
Finally, the last two instructions read:

\begin{quote}
    \textit{6. Walk along Hammer Straße for around 700 meters. You'll pass a few residential and commercial buildings.\\}
    \textit{7. Look for Hafenweg on your left. Turn left onto Hafenweg.}
\end{quote}

Again, these instructions do not appear viable, as the map context shown in Figure 1.2 does not show a visible connection between the two streets.
Whether there are ``residential and commercial buildings'' along the Hammer Straße is irrelevant, although it may as well be true.
In total, GPT-4o suggested several steps that may be difficult to follow in practice when compared against the map context shown in Figure 1.2.

Preliminary, informal tests discussed during thesis supervision suggested similar shortcomings in contemporary LLM navigation in urban environments.
In line with these preliminary observations, the routing task shown above illustrates that such issues may occur in recent LLMs such as GPT-4o.
In the remainder of this thesis, we will evaluate this question systematically across multiple models and geographic areas.

% What have observations like these led us to think about?

These observations motivate a systematic evaluation of LLMs and their abilities in spatial reasoning, and whether their performance can be improved using qualitative geographic context.
We thus conduct a series of experiments to evaluate the navigation capabilities of select LLMs across multiple geographic areas, both with and without the inclusion of qualitative geographic context.
In order to gain a clear picture of the current state of research in this area, we must first review the relevant literature.

% Research Overview & Research Gap

\section{Research Overview}

Within the scope of our literature review for this thesis, we did not identify prior work explicitly evaluating the impact of qualitative geographic context on LLM navigation performance.
Without identified prior research explicitly addressing this issue, the viability of LLMs as natural language interfaces for navigation tasks remains insufficiently characterized in the existing body of research.
While previous research has addressed the general capabilities of LLMs in reasoning tasks, it cannot automatically be assumed that these capabilities translate to strong navigation performance in real world scenarios.
Many potential uses for LLMs in GIScience applications have been theorized, among those the possibility to act as natural language interfaces for spatial queries such as navigation.
Several benchmarks indicate that LLMs struggle with planning- and navigation-related tasks.

This research alone is however not sufficient to fully dismiss the idea of using LLMs for navigation.
If research could identify techniques that substantially improve LLM navigation performance, the idea of their usage in this area could become practical in the future.
Even though one study alone will not be sufficient to fully explore this area and answer all open questions, it could provide first evidence on the viability of such techniques and serve as a benchmark for future research.

Context enrichment is widely used to provide task-relevant external information to LLMs.
Although this thesis does not test this explanation, one possible reason for poor navigation performance in LLMs could be such a blind spot in the training data, in this case regarding topological data suitable for navigation.

Any attempts to address the issue of poor LLM navigation performance using context enrichment must however answer the initial question of how to best provide geographic context in a way that is suitable for LLMs.
Since LLMs are primarily trained on text corpora, we hypothesize that the geographic context should be present in the form of text as well.
Many sources provide openly available spatial data online.
Much of this data is however present in machine-readable formats, rather than in natural language.

A solution to address this related issue could be to utilize a framework for qualitatively describing topological data such as the dipole calculus.
The dipole calculus is a framework that has been discussed in applications such as robotics.
In the framework, purely qualitative relations are defined to describe topological relations in an elegant manner.
These relations could be translated into natural language statements, suitable for LLM context enrichment.

In this study we therefore aim to show first empirical evidence on whether this technique of improving LLM navigation using qualitative geographic context could prove viable.
As mentioned, this research will address a gap in the existing literature, but will not be sufficient to fully explore the area.
The testing framework we establish can also be utilized by future researchers studying additional ways of improving LLM navigation performance.
In the end, this study will serve as a first step on the long-term path towards LLM usage in navigation tasks.

% Research Hypothesis

\section{Research Hypothesis}

Drawing from the current state of research, we propose that context enrichment using qualitative geographic relations may have a positive effect on LLM navigation performance.
Context enrichment has been shown to make LLM responses less prone to hallucinations in other domains, while dipole relations provide a proven framework for qualitatively describing topological data.
Combined, these techniques may aid the LLM in generating more accurate navigation responses.
We thus formulate the following research hypothesis ($\mathcal{H}_1$): 
\begin{quote} 
    \textit{The inclusion of qualitative geographic context in the form of natural language dipole relations substantially improves the navigation performance of large language models.}
\end{quote}
The alternative hypothesis can thus be formulated as follows ($\mathcal{H}_0$):
\begin{quote} 
    \textit{The inclusion of qualitative geographic context in the form of natural language dipole relations does not substantially improve the navigation performance of large language models.}
\end{quote}
A substantial improvement in this context is defined as a practically meaningful increase in success rate (in percentage points) relative to the control condition.
In the following chapters, we will describe the methods used to test our hypothesis and present the results of our experiments.
The thesis will then conclude with a discussion of the findings and their meaning for future research in this area.