%!TEX root = ../thesis.tex
\chapter{Introduction}
\label{ch:01intro}

In this chapter, we introduce the research topic of navigation using large language models.
After briefly discussing the current capabilities of LLMs in navigation tasks, we review the existing body of research in this emergent area.
We then identify the research gap we aim to adress in this thesis and formulate a corresponding research hypothesis.

\section{The Current State of Navigation using LLMs}

In practice, users rely on large language models for an increasing number of tasks.
Starting out as the latest advancements in natural language processing, LLMs have quickly found many other applications such as computer code generation in software development or more creative uses such as image and video synthesis.
Although earlier models possessed were limited in these areas, more recent models such as GPT-4o and Gemini 2.5 Pro have demonstrated impressive performance due to their emerging capabilities such as reasoning.
As demonstrated in the depiction below, these models can also be confronted with navigational tasks like route planning:

\begin{figure}[h!]
    \centering
    \includegraphics[width=0.8\textwidth]{chapter/PastedGraphic-2.png} % Adjust width as needed
    \caption{GPT-4o response to a simple route planning task.}
    \label{fig:pasted_graphic}
\end{figure}

Although alternative route planning solutions exist, this example illustrates the potential of using LLMs as purely text-based interfaces for navigational tasks, as compared to traditional map-based implementations such as Google Maps or OpenStreetMap.
For this concept to be viable in practice however, LLMs must provide accurate and reliable responses, as mistakes in navigation can lead to significant problems in many scenarios.
Investigation on the reliability of LLMs in this regard have already been conducted, revealing a concerning tendency to provide infactual information when confronted with route planning scenarios:

\begin{figure}[h!]
    \centering
    \includegraphics[width=0.8\textwidth]{chapter/PastedGraphic-3.png} % Adjust width as needed
    \caption{The correct route for the initial task (green) compared to the GPT-4o response (red).}
    \label{fig:pasted_graphic}
\end{figure}

Even though the model response shown in Figure 1.1 appears structurally sound and plausible at first glance, upon further inspection the example demonstrates quite well that LLMs can fail to give accurate answers to navigational tasks when compared to the ground truth (Figure 1.2).
In this case, GPT-4o suggested several wrong and disconnected directions, a striking contrast compared to the actual route.
Additionaly, traversal of a street was suggested that doesn't exist in the investigated city at the time of writing (Willy-Brandt-Allee).

Observations like these have left room for research systematically studying the shortcomings of contemporary LLMs in spatial reasoning, and if the inclusion of qualitative geographic context in the form of natural language dipole relations can significantly improve their performance in this regard. 
We thus conduct a series of experiments to evaluate the navigation capabilities of select LLMs across multiple geographic areas, both with and without the inclusion of qualitative geographic context.
In order to gain a clear picture of the current state of research in this area, we must first review the relevant literature.


\section{Research Gap}

We identify a gap in the current literature regarding the use of qualitative geographic context to enhance LLM navigation performance.
Without further research, any verdict on the viability of LLMs as natural language interfaces for navigation tasks will remain inconclusive. 
Previous research has adressed a) the general capabilities of LLMs in reasoning tasks, b) their potential use in GIScience applications as well as c) their shortcomings in navigation tasks.
This research is not sufficient to fully dismiss the idea of using LLMs for navigation, as few attempts have been made to improve their navigation performance.
Context enrichment has been shown to be a promising technique to reduce hallucinations in LLM responses, which often occur when they are confronted with spatial reasoning problems.
Further, the dipole calculus has been shown to be a valid qualitative representation of topological data.
Therefore, in ths study, we aim to quantify the impact of context enrichment using dipole relations on LLM navigation performance to provide first evidence on whether this technique could prove viable.
Additionaly, our testing framework may be used as a benchmark for future research utilizing different approaches in this area.
This way, we can thorougly study to which extent LLM navigation can be improved and reach a conclusion backed by empirical evidence.

\section{Research Hypothesis}

Drawing from the current state of research, we propose that context enrichment using qualitative geographic relations may have a positive effect on LLM navigation performance.
Context enrichment has been shown to make LLM responses less prone to hallucinations in other domains, while dipole relations provide a proven framework for qualitatively describing topological data.
Combined, these techniques should aid the LLM and allow it to provide more accurate answer to navigational tasks.
We thus formulate the following research hypothesis H1: 
\begin{quote} 
    \textit{The inclusion of qualitative geographic context in the form of natural language dipole relations significantly improves the navigation performance of large language models.}
\end{quote}
The alternative hypothesis can thus be formmulated as follows H0:
\begin{quote} 
    \textit{The inclusion of qualitative geographic context in the form of natural language dipole relations does not significantly improve the navigation performance of large language models.}
\end{quote}
In the following chapters, we will describe the methods used to test our hypothesis and present the results of our experiments.
The thesis will then conclude with a discussion of the findings and their meaning for future research in this area.