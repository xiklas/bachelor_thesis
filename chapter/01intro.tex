%!TEX root = ../thesis.tex
\chapter{Introduction}
\label{ch:01intro}

% Chapter Introduction

In this chapter, we introduce the research topic of navigation using large language models.
After briefly discussing the current capabilities of LLMs in navigation tasks, we review the existing body of research in this emergent area.
We then identify the research gap we aim to adress in this thesis and formulate a corresponding research hypothesis.

% The Current State of Navigation using LLMs

\section{The Current State of Navigation using LLMs}

% Introduction to Research Topic

Today, users rely on large language models for an increasing number of tasks.
There are practically no limits to prompting an LLM, which has led to some unexpected revelations.
Starting out as the latest advancements in natural language processing, LLMs have thus quickly found many other applications such as computer code generation in software development or more creative uses such as image and video synthesis.
Although earlier models possessed were limited in these areas, more recent models such as GPT-4o and Gemini 2.5 Pro have demonstrated impressive performance due to their reasoning capabilities.

% Example Figure of User Navigation Task Reply using ChatGPT 4o

\begin{figure}[h!]
    \centering
    \includegraphics[width=0.8\textwidth]{chapter/PastedGraphic-2.png} % Adjust width as needed
    \caption{GPT-4o response to a simple route planning task.}
    \label{fig:pasted_graphic}
\end{figure}

% Why would users want to use LLMs for navigation?

Although alternative route planning solutions exist, this example illustrates the potential of using LLMs as purely text-based interfaces for navigational tasks, as compared to traditional map-based implementations such as Google Maps or OpenStreetMap.
For this concept to be viable in practice however, LLMs must provide accurate and reliable responses, as mistakes in navigation can lead to significant problems in many scenarios.
Investigation on the reliability of LLMs in this regard have already been conducted, revealing a concerning tendency to provide infactual information when confronted with route planning scenarios:

% Map with Correct Route vs. GPT-4o Response

\begin{figure}[h!]
    \centering
    \includegraphics[width=0.8\textwidth]{chapter/PastedGraphic-3.png} % Adjust width as needed
    \caption{The correct route for the initial task (green) compared to the GPT-4o response (red).}
    \label{fig:pasted_graphic}
\end{figure}

% What is wrong with the LLM response?

Even though the model response shown in Figure 1.1 appears structurally sound and plausible at first glance, upon further inspection the example demonstrates quite well that LLMs can fail to give accurate answers to navigational tasks when compared to the ground truth (Figure 1.2).
In this case, GPT-4o suggested several wrong and disconnected directions, a striking contrast compared to the actual route.
Additionaly, traversal of a street was suggested that doesn't exist in the investigated city at the time of writing (Willy-Brandt-Allee).

% What have observations like these led us to think about?

Observations like these have left room for research systematically studying the shortcomings of contemporary LLMs in spatial reasoning, and if the inclusion of qualitative geographic context in the form of natural language dipole relations can significantly improve their performance in this regard. 
We thus conduct a series of experiments to evaluate the navigation capabilities of select LLMs across multiple geographic areas, both with and without the inclusion of qualitative geographic context.
In order to gain a clear picture of the current state of research in this area, we must first review the relevant literature.

% Research Overview & Research Gap

\section{Research Overview}

At the moment, no literature explores the impact of qualitative geographic context on navigation performance in LLMs.
Without any research, a clear verdict on the viability of LLMs as natural language interfaces for navigation tasks will remain inconclusive. 
While previous research has adressed the general capabilities of LLMs in reasoning tasks, it can not automatically be assumed that these capabilities translate to strong navigation performance in real world scenarios.
Many potential uses for LLMs in GIScience applications have been theorized, among those the possibility to act as natural language interfaces for spatial queries such as navigation.
The existing research regarding LLM navigation performance however paints a clear picture: Currently, LLMs are not able to perform navigation tasks reliably, and thus this potential use case remains unfulfilled.

This research alone is however not sufficient to fully dismiss the idea of using LLMs for navigation.
Few attempts have been made to improve their performance in this regard.
If research could identify techniques that significantly improve LLM navigation performance, the idea of their usage in this area could become practical in the future.
Even thoough one study alone will not be sufficient to fully explore this area and answer all open questions, it could provide first evidence on the viability of such techniques and serve as a benchmark for future research.

One technique that has shown promise in reducing shortcomings such as hallucinations in other areas where LLMs are applied is context enrichment.
By providing additional contex to the model, blindspots in LLM training data such as recent events, niche topics or internal company data can be adressed by this method.
It seems likely, although unproven, that a similar blindspot could be manifest in the area of topological data required for urban navigation tasks.

Any attempts to adress the issue of poor LLM navigation performance using context enrichment must however answer the initial question of how to best provide geographic context in a way that is suitable for LLMs.
Since LLMs mostly consume text data during training, we hypothesize that the geographic context should be present in the form of text as well.
Fortunately, spatial data is available in vast quantities on the internet for free.
Unfortunately, this data is usually not in a text based natural-language format.

A solution to adress this tangential issue could be to utilize a framework for qualitatively describing topological data such as the dipole calculus.
The dipole calculus is a established framework that has been used in applications such as robotics.
In the framework, purely qualitative relations are defined to describe topological relations in an elegant manner.
These relations could be translated into natural language statements, suitable for LLM context enrichment.

In this study we therefore aim to show first empirical evidence on whether this technique of improving LLM navigation using qualitative geographic context could prove viable.
As mentioned, this research will adress a gap in the existing literature, but will not be sufficient to fully explore the area.
The testing framework we establish can also be utilized by future researchers studying additional ways of improving LLM navigation performance.
In the end, this study will serve as a first step on the long-term path towards LLM usage in navigation tasks.

% Research Hypothesis

\section{Research Hypothesis}

Drawing from the current state of research, we propose that context enrichment using qualitative geographic relations may have a positive effect on LLM navigation performance.
Context enrichment has been shown to make LLM responses less prone to hallucinations in other domains, while dipole relations provide a proven framework for qualitatively describing topological data.
Combined, these techniques should aid the LLM and allow it to provide more accurate answer to navigational tasks.
We thus formulate the following research hypothesis H1: 
\begin{quote} 
    \textit{The inclusion of qualitative geographic context in the form of natural language dipole relations significantly improves the navigation performance of large language models.}
\end{quote}
The alternative hypothesis can thus be formmulated as follows H0:
\begin{quote} 
    \textit{The inclusion of qualitative geographic context in the form of natural language dipole relations does not significantly improve the navigation performance of large language models.}
\end{quote}
In the following chapters, we will describe the methods used to test our hypothesis and present the results of our experiments.
The thesis will then conclude with a discussion of the findings and their meaning for future research in this area.