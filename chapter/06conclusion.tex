%!TEX root = ../thesis.tex
\chapter{Conclusion and Outlook}
\label{ch:06conclusion}

In this study, we have explored the effects of qualitative geographic context on LLM navigation performance.
Our research hypothesis stated that giving language models access to qualitative geographic context would improve their navigation capabilities.
To test this hypothesis, we designed a series of experiments comparing LLM performance on navigation tasks in two independent conditions:
one with qualitative geographic context and one without.
We utilized randomly generated navigation tasks in our two chosen test areas (Münster and Hamburg) and let both conditions generate answers to each task.
The answers were then evaluated manually based on a correctness criterion.

The results of our experiments show that LLMs perform poorly in navigation tasks without qualitative geographic context, answering 0 out of 120 tasks correctly, corresponding to a success rate of 0\%.
In contrast, with additional geographic context, the models achieved a substantially higher performance, answering 75 out of 120 tasks correctly, corresponding to a success rate of 62.5\%.
These findings support our initial research hypothesis:
All three tested LLMs (GPT-4o, Gemini 2.5 Pro, Claude Sonnet 4.5) demonstrated a higher success rate on navigation tasks when provided with qualitative geographic context.

Our setup involved just two specific urban areas from the cities of Münster and Hamburg.
In addition, out of many available LLMs, we only tested three models.
In future research, these limitations could be addressed by including more variety both in the tested geographic areas and in the selected LLMs.
Further, our evaluation was based on human labeling of a large number of generated answers.
This time-consuming process could be improved in future studies by developing methods for automated evaluation of navigation task answers.
Thus, while our approach showed promising first results, these limitations can hopefully be addressed in future research.
Overall, this study establishes a foundation for context-enriched LLM navigation and motivates continued research in this domain.